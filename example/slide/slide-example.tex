\documentclass[10.5pt, a4, dvipdfmx, landscape]{article}
\usepackage{slide}

\title{確率的に生成されたデータに対するパーシステントホモロジーの推定能力について}

\begin{document}

\maketitle

\sectionframe{ホモロジー群$H_n \left( K \right)$の構成について}

\begin{frame}
  \frametitle{sect}
  ここで,$0 \leq i \leq k$に対して,写像$\varepsilon_i: \Delta^{k - 1} \to \Delta^k$を
  \begin{align*}
    \varepsilon_0 \left( x_1, \cdots, x_{k - 1} \right) & = \left( 1, x_1, \cdots, x_{k - 1} \right)                                               \\
    \varepsilon_i \left( x_1, \cdots, x_{k - 1} \right) & = \left( x_1, \cdots, x_i, x_i, \cdots, x_{k - 1} \right) \quad \left( 0 < i < k \right) \\
    \varepsilon_k \left( x_1, \cdots, x_{k - 1} \right) & = \left( x_1, \cdots, x_{k - 1}, 0 \right)
  \end{align*}
  により定義するとき,$\varepsilon_i$を$\varepsilon_i$上のAffine写像という。

  \begin{dfn}
    ここで,$0 \leq i \leq k$に対して,写像$\varepsilon_i: \Delta^{k - 1} \to \Delta^k$を
    \begin{align*}
      \varepsilon_0 \left( x_1, \cdots, x_{k - 1} \right) & = \left( 1, x_1, \cdots, x_{k - 1} \right)                                               \\
      \varepsilon_i \left( x_1, \cdots, x_{k - 1} \right) & = \left( x_1, \cdots, x_i, x_i, \cdots, x_{k - 1} \right) \quad \left( 0 < i < k \right) \\
      \varepsilon_k \left( x_1, \cdots, x_{k - 1} \right) & = \left( x_1, \cdots, x_{k - 1}, 0 \right)
    \end{align*}
    により定義するとき,$\varepsilon_i$を$\varepsilon_i$上のAffine写像という。
  \end{dfn}
\end{frame}

\begin{frame}
  $0 \leq i \leq k$に対して,写像$\varepsilon_i: \Delta^{k - 1} \to \Delta^k$を
  \begin{align*}
    \varepsilon_0 \left( x_1, \cdots, x_{k - 1} \right) & = \left( 1, x_1, \cdots, x_{k - 1} \right)                                               \\
    \varepsilon_i \left( x_1, \cdots, x_{k - 1} \right) & = \left( x_1, \cdots, x_i, x_i, \cdots, x_{k - 1} \right) \quad \left( 0 < i < k \right) \\
    \varepsilon_k \left( x_1, \cdots, x_{k - 1} \right) & = \left( x_1, \cdots, x_{k - 1}, 0 \right)
  \end{align*}
  により定義するとき,により定義するとき,により定義するとき,により定義するとき,により定義するとき,により定義するとき,により定義するとき,により定義するとき,により定義するとき,により定義するとき,$\varepsilon_i$を$\varepsilon_i$上のAffine写像という\cite{tst}。
  Afra Zomorodian and Gunnar Carlsson. Computing persistent homol-
  ogy. In Proceedings of the twentieth annual symposium on Computa-
  tional geometry, pages 347-356, 2004
  \begin{itemize}
    \item Afra Zomorodian and Gunnar Carlsson. Computing persistent homol-
          ogy. In Proceedings of the twentieth annual symposium on Computa-
          tional geometry, pages 347-356, 2004
    \item 特定の処理をしたデータのホモロジー群を考えることにより,データに潜む幾何学的特徴を抽出する。
    \item 構造定理\cite{zomorodian2004computing}と安定性定理\cite{cohen2005stability}という二つの強力な数学的背景により,データ解析技術としての信頼性を担保している。
    \item 材料科学\cite{nakamura2015persistent}\cite{hiraoka2016hierarchical}や画像解析\cite{tanabe2021homological},グラフ理論\cite{chan2013topology}など幅広い研究分野で応用されている。
  \end{itemize}
\end{frame}

\begin{referenceframe}
  \bibliographystyle{unsrt}
  \bibliography{papers}
\end{referenceframe}

\end{document}