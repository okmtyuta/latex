\documentclass[dvipdfmx,11pt,notheorems]{beamer}
\usepackage{slide}

\title{Example: Slide style}
\author{OKAMOTO Yuta}
\institute{Division of Informatics, Graduate School of Informatics, Kyoto University}
\date{\today}

\addtobeamertemplate{footline}{\quad This is a footer}{}

\begin{document}

\begin{frame}
  \titlepage
\end{frame}

\begin{frame}\framesection{Education and research}
  \begin{itemize}
    \item Bachelor's degree. Department of Mathematics, Faculty of Science, University of Tokyo : 2020 April - 2024 March.
          \begin{itemize}
            \item My advisor was MATSUI Chihiro. According to Matsui sensei, her lab is considered to be in the legitimate lineage of KUBO Ryogo, a highly renowned physicist.
          \end{itemize}
    \item Master's degree. Division of Informatics, Graduate School of Informatics, Kyoto University: 2024 April - Current.
          \begin{itemize}
            \item My advisor is TANAKA Toshiyuki. I don't know much about Tanaka sensei, but it seems he is quite well-known in the machine learning community.
          \end{itemize}
  \end{itemize}
\end{frame}

\begin{frame}\framesubsection{Education and research}
  \begin{columns}[c]
    \begin{column}{0.6\textwidth}
      During my undergraduate years, I majored in mathematical physics. Specifically, I focused on formulating quantum mechanics using operators on Hilbert spaces, based on John von Neumann's ``Mathematical Foundations of Quantum Mechanics.''
    \end{column}
    \begin{column}{0.4\textwidth}
      \begin{center}
        \includegraphics[width=0.8\columnwidth]{global_okmtyuta.png}
      \end{center}
    \end{column}
  \end{columns}
\end{frame}

\begin{frame}\frametitle{Education and research: Bachelor's degree}
  \begin{dfn}
    \begin{itemize}
      \item According to von Neumann's definition, a Hilbert space is a complete and separable inner product space.
      \item An operator on a Hilbert space $H$ is a self-homomorphism on $H$.
    \end{itemize}
  \end{dfn}

  \begin{eg}
    Sequence spaces $l^2$ and function spaces $L^2$ are well-known examples of Hilbert spaces.
  \end{eg}
\end{frame}

\begin{frame}\frametitle{Education and research: Master's degree}
  Currently, I am working on two research projects.

  \begin{itemize}
    \item The probabilistic properties of persistent homology
    \item Prediction tasks using natural language models for proteins
  \end{itemize}
\end{frame}

\section{Persistent homology}
\begin{frame}\frametitle{Persistent homology: homology}
  Homology theory is a method for algebraically determining the geometric features of a topological space (intuitively, the number of connected components, holes, and cavities).
\end{frame}

\begin{frame}\frametitle{Persistent homology: definition}
  Persistent homology is a data analysis method that applies homology theory, aiming to extract the geometric features of data by analyzing the homology that changes over time.
\end{frame}

\section{Persistent homology: research problem}
\begin{frame}\frametitle{Persistent homology}
  I am analyzing the probabilistic structure of the persistent homology associated with data $D$ that is independently and identically generated according to a probability measure $P$ from various perspectives.
\end{frame}

\section{Protein LLM}
\begin{frame}\frametitle{Protein LLM: introduction}
  \alert{Homology theory} is a method for algebraically determining the geometric features of a topological space (intuitively, the number of connected components, holes, and cavities).
\end{frame}

\section{As a Software Engineer}
\begin{frame}\frametitle{As a Software Engineer}
  Since entering university, I have been studying programming and have worked as a software engineer for a long time.

  Recently, I have been focusing on building animations using modern CSS with minimal reliance on JavaScript and implementing a Markdown parser using Rust.

  I started my own business as a sole proprietor, aiming to strike it rich. Just aiming, though.
\end{frame}

\section{Today's Goal}
\begin{frame}\frametitle{Today's Goal}
  Today, I hope to spark everyone's interest in mathematics and computers, even if just a little bit.

  Personally, I was inspired to pursue mathematics and physics after watching the drama ``Galileo.'' If I can be that ``Galileo'' for everyone here today, it would be incredibly rewarding.
\end{frame}

\section{Today's Goal}
\begin{frame}\frametitle{Today's Goal}
  I welcome any questions related to studying or research, no matter how trivial they may seem. Of course, there might be many questions I can't answer due to my own lack of knowledge, but I will do my best to provide answers whenever possible.

  \alert{\underline{Please refrain from asking questions that might take me to position talk.}}
\end{frame}

\end{document}